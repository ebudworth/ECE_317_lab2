% ECE317 Project 01
% Kai Brooks
% Fall 2019

% Document settings ------------------------------------------------
\documentclass[a4paper,12pt]{article}

\newcommand{\figOverlay}{\put(34,10){\color{black!50} \figWatermark}} % Figure overlay settings
\newcommand{\figWatermark}{}%\small Brooks \today} 		% Figure overlay text
\newcommand{\figHere}{\begin{overpic}[percent,scale=0.34]}	% Settings for all figures

% Packages ------------------------------------------------

\usepackage[USenglish]{babel} 	% American English
\usepackage{blindtext}			% Generate latin crap
\usepackage[yyyymmdd]{datetime} % Sets date format to ISO 8601 standard
\renewcommand{\dateseparator}{-}% Sets date format to ISO 8601 standard

\usepackage{graphicx}			% Image importing and display
\graphicspath{ {images/} }		% Path to image folder
\usepackage{xcolor}				% Allows normal color words
\usepackage{color, colortbl}


\usepackage{float}				% Adds 'H' for figure placement location
\usepackage{enumitem}			% Use for QandA environment
\usepackage{booktabs}			% Merging columns in tables

\usepackage[nostamp]{draftwatermark}	% [nostamp][firstpage]
\SetWatermarkText{DRAFT}
\SetWatermarkColor{red!50}
\SetWatermarkScale{3}


\usepackage{overpic}				% Puts text over figures
\usepackage[american]{circuitikz}	% American-style circuit diagrams

\usepackage{amsmath}				% Multi-line equations
\usepackage{caption}				% Equation caption formatting
\usepackage{physics}				% Easier derivatives
\usepackage{gensymb}				% Enable \degree for degree symbol

\usepackage{array}					% Used for centering tabular data
\newcolumntype{M}[1]{>{\centering\arraybackslash}p{#1}} % The actual centered column format

\usepackage{listings} %For code in appendix

\definecolor{mymauve}{rgb}{0.58,0,0.82}
\definecolor{mygreen}{rgb}{0,0.6,0}
\definecolor{mygray}{rgb}{0.5,0.5,0.5}
\definecolor{ltgray}{rgb}{0.937, 0.937, 0.956}	% Divide standard RGB values by 255 for some reason 

% PSU colors
\definecolor{PSUgreen}{RGB}{106,127,16}
\definecolor{PSUltgreen}{RGB}{168,180,0}
\definecolor{PSUblue}{RGB}{0,117,154}
\definecolor{PSUltblue}{RGB}{161,216,224}
\definecolor{PSUgray}{RGB}{71,67,52}
\definecolor{PSUbrown}{RGB}{96,53,29}
\definecolor{PSUsienna}{RGB}{163,63,31}
\definecolor{PSUred}{RGB}{210,73,42}
\definecolor{PSUorange}{RGB}{220,155,50}
\definecolor{PSUyellow}{RGB}{230,220,143}
\definecolor{PSUtan}{RGB}{232,221,162}
\definecolor{PSUpurple}{RGB}{101,3,96}


\newenvironment{QandA}
	{\begin{enumerate}[label=\arabic*.]\sl} % Use slanted question text and Arabic numerals
  {\end{enumerate}}
\newenvironment{answered}{\par\normalfont}{} % Paragraph break and use normal font

% fancy header / footer lines
\usepackage{fancyhdr}% http://ctan.org/pkg/fancyhdr
\pagestyle{fancy}% Change page style to fancy
\fancyhf{}% Clear header/footer
\fancyhead[L]{\textcolor{PSUgray}{ECE317}}
\fancyhead[R]{\textcolor{PSUgray}{Project 01}}
\fancyfoot[L]{\textcolor{PSUgray}{Brooks}}
\fancyfoot[R]{\textcolor{PSUgray}{\thepage}}
\renewcommand{\headrulewidth}{0.4pt}% Default \headrulewidth is 0.4pt
\renewcommand{\footrulewidth}{0.4pt}% Default \footrulewidth is 0pt


% Title Page ------------------------------------------------
\begin{document}
\lstset { %Formatting for code in appendix
  language=Matlab,
  basicstyle=\footnotesize\ttfamily,
  numbers=left,
  stepnumber=1,
  showstringspaces=false,
  tabsize=1,
  breaklines=true,
  breakatwhitespace=false,
  stringstyle=\color{mymauve},
  keywordstyle=\color{blue},
  commentstyle=\color{mygreen}, 
}

\begin{titlepage}
	\begin{center}
		\vspace*{1cm}

		\huge\textsc{Project 01}

		\vspace{0.5cm}
		\small\textsc{ECE 317}
		
		\vspace{1.5cm}
		\normalsize Kai Brooks
		
		\vspace{0.5cm}
		Lab TA: N/A
		
		\vfill
		\vspace{0.8cm}
		
		\includegraphics[width=0.5\textwidth]{images/psulogo_horiz_msword.tif}
		
		\vspace{0.5cm}
		Electrical and Computer Engineering\\
		Portland State University\\
		\today
		 
	\end{center}
\end{titlepage}

% Table of contents ------------------------------------------------
\newpage
\tableofcontents


% Begin paper ------------------------------------------------
\newpage
\pagenumbering{arabic}

\section{Overview}
The purpose of this lab is to understand the fundamental usage of PECS and determine how variations in switching frequency, clocks, and pulse width modulator affect various parts of electric circuits.

\section{Circuit 1}
	\subsection{Circuit Diagram}
	\begin{figure}[H]	 		
			\centering
	  	\label{fig:}
	  	\figHere{images/01_circuit01.png} \figOverlay
	  	\end{overpic}
	  	\caption{Circuit}
	\end{figure}
	
	\begin{figure}[H]	 		
		\centering
	  	\label{fig:}
	  	\figHere{images/01_parameters01.png} \figOverlay
	  	\end{overpic}
	  	\caption{Simulation Parameters}
	\end{figure}
	
	\subsection{Output}
	\begin{figure}[H]	 		
			\centering
	  	\label{fig:}
	  	\figHere{images/01_waveform01.png} \figOverlay
	  	\end{overpic}
	  	\caption{Full waveform}
	\end{figure}
	
	\begin{figure}[H]	 		
			\centering
	  	\label{fig:}
	  	\figHere{images/01_zoom01.png} \figOverlay
	  	\end{overpic}
	  	\caption{Zoomed waveform}
	\end{figure}
	
	\begin{figure}[H]	 		
		\centering
	  	\label{fig:}
	  	\figHere{images/01_comp01.png} \figOverlay
	  	\end{overpic}
	  	\caption{Comparative waveform}
	\end{figure}
	
	\subsection{Analysis}
	\begin{table}[H]
	\centering
		\begin{tabular}{|M{.25\textwidth}|M{.25\textwidth}|M{.25\textwidth}|M{.25\textwidth}|} % Col width
		\hline
		\textbf{Peak Amplitude} & \textbf{Period} & \textbf{Pulse Width} & \textbf{Duty Ratio} \\ \hline
		$10v$ & 10$\mu$s & 4$\mu$s & 0.4 \\ \hline
		\end{tabular}						
		\caption{Circuit 1 calculations}	
	\end{table}
	f
	\subsection*{Question}	
	\textit{Taking the above plots into consideration, explain why you would expect to get the steady state value you found above?}
	
	\subsection*{Answer}
	\begin{equation}
	V_{SS} = V_{peak} \cdot \;Duty\; Ratio
	\end{equation}
	It is expected to get the steady state above as the peak voltage is $10v$ and the duty ratio is 0.4.
	
\section{Circuit 2}
	\subsection{Circuit Diagram}
	\begin{figure}[H]	 		
		\centering
	  	\label{fig:}
	  	\figHere{images/01_circuit02.png} \figOverlay
	  	\end{overpic}
	  	\caption{Sawtooth wave circuit}
	\end{figure}
	
	\begin{figure}[H]	 		
		\centering
	  	\label{fig:}
	  	\figHere{images/01_parameters02.png} \figOverlay
	  	\end{overpic}
	  	\caption{Simulation Parameters}
	\end{figure}
	
	\subsection{Output}
	\begin{figure}[H]	 		
			\centering
	  	\label{fig:}
	  	\figHere{images/01_comp02.png} \figOverlay
	  	\end{overpic}
	  	\caption{Comparative waveform}
	\end{figure}
	
	\subsection{Analysis}
	\begin{table}[H]
	\centering
		\begin{tabular}{|M{.5\textwidth}|M{.5\textwidth}|} % Col width
		\hline
		\textbf{Peak Amplitude} & \textbf{Period} \\ \hline
		$2v$ & 200$\mu$s \\ \hline
		\end{tabular}						
		\caption{Circuit 2 calculations}	
	\end{table}
	
	\subsection*{Question}	
	\textit{Given that $K_1 = -1$, what other factors in the circuit determines the peak amplitude and why?}
	
	\subsection*{Answer}
	\begin{equation}
	Steady\;State = Peak\;Amplitude\, \cdot \;Duty\; Ratio
	\end{equation}
	It is expected to get the steady state above as the peak voltage is $10v$ and the duty ratio is 0.4.
	
\section{Circuit 3}
	\subsection{Circuit Diagram}
	\begin{figure}[H]	 		
		\centering
	  	\label{fig:}
	  	\figHere{images/01_circuit02.png} \figOverlay
	  	\end{overpic}
	  	\caption{Circuit (adjusted)}
	\end{figure}
	
	\subsection{Analysis}
	\begin{table}[H]
	\centering
		\begin{tabular}{|M{.16\textwidth}|M{.16\textwidth}|M{.16\textwidth}|M{.16\textwidth}|M{.16\textwidth}|M{.16\textwidth}|} % Col width
		\hline
		& \textbf{Switching Frequency} & \textbf{Duty Ratio} & \textbf{Peak-to-peak Input Voltage to Filter} & \textbf{Steady State Average Output Voltage} & \textbf{Peak-to-peak Output Voltage Ripple} \\ \hline
		\textbf{Circuit 1} & 100kHz & 0.4 & 10v & 4v & 5.4mV \\ \hline
		\textbf{Circuit 3} & 40kHz & 0.5 & 10v & 5v & 3.5mV \\ \hline
		\end{tabular}						
		\caption{Comparative omnibus}
	\end{table}
	
	\subsection*{Question}	
	\textit{Explain the differences seen in the peak-to-peak ripple voltage values between Circuit 3 and Circuit 1. Are they in line with your expectations? Why?}
	
	\subsection*{Answer}
	\begin{equation}
	K_1(V_{P1})+K_3 < 0
	\end{equation}
	
	The ripple is smaller in circuit 3 because the capacitor is being charged and discharged more frequently. This results in a smaller ripple voltage as seen in the above plots.

\end{document}