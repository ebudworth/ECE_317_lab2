% ECE317 Project 02
% Kai Brooks,  Eric Budworth, Andrew Forsman
% Fall 2019

% Document settings ------------------------------------------------
\documentclass[a4paper,12pt]{article}

\newcommand{\figOverlay}{\put(34,10){\color{black!50} \figWatermark}} % Figure overlay settings
\newcommand{\figWatermark}{}%\small Brooks \today} 		% Figure overlay text
\newcommand{\figHere}{\begin{overpic}[percent,scale=0.34]}	% Settings for all figures

% Packages ------------------------------------------------

\usepackage[USenglish]{babel} 	% American English
\usepackage{blindtext}			% Generate latin crap
\usepackage[yyyymmdd]{datetime} % Sets date format to ISO 8601 standard
\renewcommand{\dateseparator}{-}% Sets date format to ISO 8601 standard

\usepackage{graphicx}			% Image importing and display
\graphicspath{ {images/} }		% Path to image folder
\usepackage{xcolor}				% Allows normal color words
\usepackage{color, colortbl}


\usepackage{float}				% Adds 'H' for figure placement location
\usepackage{enumitem}			% Use for QandA environment
\usepackage{booktabs}			% Merging columns in tables

\usepackage[firstpage]{draftwatermark}	% use [nostamp] when finished, [firstpage] otherwise
\SetWatermarkText{DRAFT}
\SetWatermarkColor{red!50}
\SetWatermarkScale{3}




\usepackage{overpic}				% Puts text over figures
\usepackage[american]{circuitikz}	% American-style circuit diagrams

\usepackage{amsmath}				% Multi-line equations
\usepackage{caption}				% Equation caption formatting
\usepackage{physics}				% Easier derivatives
\usepackage{gensymb}				% Enable \degree for degree symbol
\usepackage{siunitx}				% SI units


\usepackage{array}					% Used for centering tabular data
\newcolumntype{M}[1]{>{\centering\arraybackslash}p{#1}} % The actual centered column format

\usepackage{listings} %For code in appendix

\definecolor{mymauve}{rgb}{0.58,0,0.82}
\definecolor{mygreen}{rgb}{0,0.6,0}
\definecolor{mygray}{rgb}{0.5,0.5,0.5}
\definecolor{ltgray}{rgb}{0.937, 0.937, 0.956}	% Divide standard RGB values by 255 for some reason 

% PSU colors
\definecolor{PSUgreen}{RGB}{106,127,16}
\definecolor{PSUltgreen}{RGB}{168,180,0}
\definecolor{PSUblue}{RGB}{0,117,154}
\definecolor{PSUltblue}{RGB}{161,216,224}
\definecolor{PSUgray}{RGB}{71,67,52}
\definecolor{PSUbrown}{RGB}{96,53,29}
\definecolor{PSUsienna}{RGB}{163,63,31}
\definecolor{PSUred}{RGB}{210,73,42}
\definecolor{PSUorange}{RGB}{220,155,50}
\definecolor{PSUyellow}{RGB}{230,220,143}
\definecolor{PSUtan}{RGB}{232,221,162}
\definecolor{PSUpurple}{RGB}{101,3,96}


\newenvironment{QandA}
	{\begin{enumerate}[label=\arabic*.]\sl} % Use slanted question text and Arabic numerals
  {\end{enumerate}}
\newenvironment{answered}{\par\normalfont}{} % Paragraph break and use normal font

% fancy header / footer lines
\usepackage{fancyhdr}% http://ctan.org/pkg/fancyhdr
\pagestyle{fancy}% Change page style to fancy
\fancyhf{}% Clear header/footer
\fancyhead[L]{\textcolor{PSUgray}{ECE317}}
\fancyhead[R]{\textcolor{PSUgray}{Project 02}}
\fancyfoot[L]{\textcolor{PSUgray}{Brooks, Budworth, Forsman}}
\fancyfoot[R]{\textcolor{PSUgray}{\thepage}}
\renewcommand{\headrulewidth}{0.4pt}% Default \headrulewidth is 0.4pt
\renewcommand{\footrulewidth}{0.4pt}% Default \footrulewidth is 0pt


% Title Page ------------------------------------------------
\begin{document}
\lstset { %Formatting for code in appendix
  language=Matlab,
  basicstyle=\footnotesize\ttfamily,
  numbers=left,
  stepnumber=1,
  showstringspaces=false,
  tabsize=1,
  breaklines=true,
  breakatwhitespace=false,
  stringstyle=\color{mymauve},
  keywordstyle=\color{blue},
  commentstyle=\color{mygreen}, 
}

\begin{titlepage}
	\begin{center}
		\vspace*{1cm}

		\huge\textsc{Project 02}

		\vspace{0.5cm}
		\small\textsc{ECE 317}
		
		\vspace{1.5cm}
		\normalsize Kai Brooks, Eric Budworth, Andrew Forsman
		
		\vspace{0.5cm}
		%Lab TA: N/A
		
		\vfill
		\vspace{0.8cm}
		
		\includegraphics[width=0.5\textwidth]{images/psulogo_horiz_msword.png}
		
		\vspace{0.5cm}
		Electrical and Computer Engineering\\
		Portland State University\\
		\today
		 
	\end{center}
\end{titlepage}

% Table of contents ------------------------------------------------
\newpage
\tableofcontents


% Begin paper ------------------------------------------------
\newpage
\pagenumbering{arabic}

\section{Overview}
The purpose of this lab is too build a passive LRC Bandpass filter, and evaluate its output signal in relation to its components and transform function. We began by building the given circuit in PECS, and program a step function that increases the input voltage by 10\%. We then plot the output funtion of this circuit, and view how the step function effects the output voltage. We were looking for the maximum oscillation output ($c_{max}$), the percent overshoot (\%OS), the settling time ($T_s$), damping ratio($\zeta$), the natural frequency($\omega_n$) and the gain (K). We then find the transfer function, and derive the natural frequency, the damping ratio, and the gain.

\section{Calculations}
	\subsection{PECS Simulation}

Calculations from Tasks 5 - 8:
\begin{align}
\begin{split}\label{eq:1}
	c_{max} &={} \SI{5.93085}{\volt} - \SI{5}{\volt} \\
	c_{max} &={} \SI{0.93085}{\volt}
\end{split}
\end{align}

\begin{align}
\begin{split}\label{eq:2}
	c_{final} &= \SI{5.5}{\volt} - \SI{5}{\volt} \\
	c_{final} &= \SI{0.5}{\volt}
\end{split}	
\end{align} 

$\%OS = \frac{\SI{0.93085}{\volt}-\SI{0.5}{\volt}}{\SI{0.5}{\volt}} \cdot 100$\\

$\%OS =  86.17\%$\\

$\zeta = \frac{-ln(86.17/100)}{\sqrt{\pi^2+ln^2(86.17/100)}}$\\

$\zeta = 0.047327$\\

$\SI{0.5}{\volt}*0.02 = \SI{0.01}{\volt}$\\
 
$c_{final}\pm 2\% = \SI{0.5}{\volt} \pm\SI{0.01}{\volt}$\\
 
$c_{final}'\pm 2\% = \SI{5.51}{\volt} \:\textrm{and} \: \SI{5.49}{\volt}$\\
 
$T_s = \SI{0.130084}{\second} - \SI{0.11}{\second}$\\
 
$T_s = \SI{0.0200840}{\second}$\\

$\omega_n  = \frac{4}{0.047327 \cdot 0.0200840}$\\

$\omega_n  = \SI{4208.24}{\frac{\radian}{\second}} $

$K = \frac{\SI{5.5}{\volt} - \SI{5}{\volt}}{\SI{5.5}{\volt} - \SI{5}{\volt}}$\\
 
$K = 1$

\subsection{MATLAB}

Derivation of Transfer Function, symbolically:\\

$Z_{RC} = (1/R+Cs)^{-1}$\\

$Z_{RC} = \frac{R}{sRC+1}$\\

$Z_{EQ} = Ls + Z_{RC}$\\

$Z_{EQ} = \frac{RLCs^2+Ls+R}{RCs+1}$\\

$V_{\textrm{out}} = V_{\textrm{in}} \cdot \frac{Z_{RC}}{Z_{EQ}}$\\

$\frac{V_{\textrm{out}}}{V_{\textrm{in}}} = \frac{R}{RLCs^2+Ls+R}$\\

$G(s) = \frac{1}{LCs^2+\frac{L}{R}s+1}$\\

So, we can see $a_1 = L/R$ and $a_2 = LC$.\\

$a_1 = \frac{560\mu}{25}\\$

$\;\;\;\;\: = 22.4\mu$\\

and\\

$a_2 = 560\mu \cdot100\mu$ \\

$\;\;\;\;\: = 56n$\\

To determine the values for K, $\zeta$, and $\omega_n$:\\

$\omega_n  = \frac{1}{\sqrt{a_2}}$\\

$\;\;\;\;\;= \SI{4225.8}{\frac{rad}{sec}}$\\

$\zeta = \frac{a_1}{2a_2\omega_n}$\\

$\;\;\;\;\;= \SI{0.047329}$\\



	
	% old stuff
	
\section{Circuit 2}
	\subsection{Circuit Diagram}
	\begin{figure}[H]

		\centering
	  	\label{fig:}
	  	\figHere{images/Circuit.png} \figOverlay
	  	\end{overpic}
	  	\caption{Circuit Diagram}
	\end{figure}

	\subsection{Analysis}
	
	\begin{table}[H]
	\centering
		\begin{tabular}{|M{.25\textwidth}|M{.25\textwidth}| M{.25\textwidth} | M{.25\textwidth} |} % Col width
		\hline
		\textbf{Transfer Function Parameters}  & \textbf{PECS Simulation Derived values From Tasks 6, 7, \& 8)}   & \textbf{From Transfer Function Derivation: Symbolic Form}     & \textbf{From Transfer Function Derivation: Evaluated}  \\ \hline
		$\zeta$         & 0.047327  &         $\frac{1}{2RC\omega_n}$          &  0.047329 \\ \hline
		$\omega_{n}$  & $\SI{4352.3}{\frac{\radian}{\second}}$      & $\frac{1}{\sqrt{a_2}}$    & $\SI{4225.8}{\frac{rad}{sec}}$  \\ \hline
		K           & 1     &                                   & 1  \\ \hline
		\end{tabular}						
		\caption{Matlab Simulations Results Check}	
	\end{table}
	
	
		\begin{table}[H]
	\centering
		\begin{tabular}{|M{.25\textwidth}|M{.25\textwidth}| M{.25\textwidth} | } % Col width
		\hline
		\textbf{Response Feature} & \textbf{From PECS Simulation} & \textbf{From Matlab Function \textit{stepinfo}}   \\  \hline
		\% Overshoot (\% OS)         &   86.17\%       & 83.74\%  \\ \hline
		Settling Time, \textit{$T_{s}$}  & 20.084 ms      & 18.8 ms      \\ \hline
		\end{tabular}						
		\caption{PECS versus Matlab}	
	\end{table}
	


	\section{Questions}
	After adding a resistor $r_{L}$ in series with the inductor L, state your observations. Does it (substantially) change any of the following of the transfer function: DC Gain, Undamped natural frequency, Damping Factor?
	
	
		\begin{QandA}
		
	\item DC Gain?
		\begin{answered}
				
		\end{answered}
		
	\item Undamped natural frequency?
	    \begin{answered}
	    
	    \end{answered}

    \item Damping Factor?
        \begin{answered}
        
        \end{answered}
	\end{QandA}	
	
	\section{MATLAB Code}
	\lstinputlisting[language=Matlab]{simulations/lab02.m}

\end{document}

